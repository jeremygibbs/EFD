%%% Template originaly created by Karol Kozioł (mail@karol-koziol.net) and modified for ShareLaTeX use

\documentclass[11pt]{article}

\usepackage[T1]{fontenc}
\usepackage[utf8]{inputenc}
\usepackage{graphicx}
\usepackage{xcolor}
\usepackage{float}
\usepackage{tgtermes}
\usepackage{natbib}
%\usepackage[subnum]{cases}
\usepackage[super]{nth}
\bibpunct{(}{)}{;}{a}{}{,}
\usepackage{amsmath,amssymb}
\usepackage{enumerate}
\usepackage{multicol}
\usepackage{tikz}
\usepackage[amssymb]{SIunits}
\usepackage{rotating}
\usepackage{enumitem}
\usepackage{geometry}
\geometry{total={8.5in,11in},
left=1in,right=1in,%
bindingoffset=0mm, top=1in,bottom=1in}
\usepackage[super]{nth}
\usepackage[
pdftitle={EFD: Homework 1}, 
pdfauthor={Jeremy Gibbs, University of Utah},
colorlinks=true,linkcolor=blue,urlcolor=blue,citecolor=blue,bookmarks=true,
bookmarksopenlevel=2]{hyperref}

\linespread{1.1}
\setlength{\parskip}{1em}
\setlength{\parindent}{0pt}
\newcommand{\linia}{\rule{\linewidth}{0.5pt}}

\makeatletter
\renewcommand{\maketitle}{
\begin{center}
\vspace{2ex}
{\huge \textsc{\@title}}
\vspace{1ex}
\\
\linia\\
ME EN 7710 \hfill Homework \#2 \hfill Due: March \nth{24}
\vspace{4ex}
\end{center}
}
\makeatother
%%%

% custom footers and headers
\usepackage{fancyhdr,lastpage}
\pagestyle{fancy}
\lhead{}
\chead{}
\rhead{}
%\lfoot{Assignment \textnumero{} 5}
\cfoot{}
\rfoot{Page \thepage~/~\pageref*{LastPage}}
\renewcommand{\headrulewidth}{0pt}
\renewcommand{\footrulewidth}{0pt}
%

%%%----------%%%----------%%%----------%%%----------%%%

\begin{document}

\title{Environmental Fluid Dynamics}

\maketitle

%-- Question 1 --%
\vspace{-20pt}
\paragraph{1.) Boundary Layer Profiles}~\\\\
Given the provided ballon data (\href{http://gibbs.science/efd/homework/balloon_data.txt}{http://gibbs.science/efd/homework/balloon\_data.txt}), perform the following:
\begin{enumerate}[label=(\alph*)]
	\item Plot vertical profiles of all the raw variables on a single page with clearly labeled axes. Note any interesting features.
	\item Calculate and plot the (i) temperature, (ii) potential temperature, and (iii) virtual potential temperature. Compare and discuss the potential temperature and virtual potential temperatures. Why do they differ?
	\item Compare the potential temperature calculations using the balloon-based pressure measurements with $\theta(z) \approxeq T(z)+ \Gamma z$. Calculate and plot the difference. What is the difference?
\end{enumerate}
 
%-- Question 2 --%
\paragraph{2.) The Laminar Ekman Layer Above a Rigid Surface}~\\\\
The simplified momentum equations:
\begin{align*}
-fv &= -\frac{1}{\rho}\frac{\partial p}{\partial x} + \nu\frac{\partial^2 u}{\partial z^2}\\
fu &=  -\frac{1}{\rho}\frac{\partial p}{\partial y} + \nu\frac{\partial^2 v}{\partial z^2}
\end{align*}
can be further simplified by expressing pressure gradients in terms of geostrophic velocity components as:
\begin{align}
\label{ek1}
-f(v-V_g) &= \nu\frac{\partial^2}{\partial z^2}(u-U_g)\\
\label{ek2}
f(u-U_g)  &= \nu\frac{\partial^2}{\partial z^2}(v-V_g)
\end{align}
\begin{enumerate}[label=(\alph*),topsep=-10pt]
	\item Assuming $U_g$ and $V_g$ are height independent, solve Eqs.~(\ref{ek1}) and (\ref{ek2}) subject to the following boundary conditions:
	\begin{itemize}
		\item $u(z$=$0) = v(z$=$0) = 0$
		\item $u(z\rightarrow \infty)=U_g$, $v(z\rightarrow \infty)=V_g$
	\end{itemize}
	For the final solution, orient the $x$-axis with the geostrophic wind vector (i.e., $U_g$ = $U$ and $V_g$ = $0$). The solutions should be in the form (where $a$ is the inverse Ekman depth):
	$$u = U[ 1 - e^{-az} \cos(az)]\qquad v = Ge^{-az}\sin(az)$$
	\item Plot your solution as a hodograph and as vertical profiles of $u$ and $v$.
	\item Please explain the phenomena of Ekman pumping.
\end{enumerate}

\end{document}