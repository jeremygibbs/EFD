%%% Template originaly created by Karol Kozioł (mail@karol-koziol.net) and modified for ShareLaTeX use

\documentclass[11pt]{article}
\usepackage{esvect}
\usepackage[T1]{fontenc}
\usepackage[utf8]{inputenc}
\usepackage{graphicx}
\usepackage{xcolor}
\usepackage{float}
\usepackage{tabto}
\usepackage{bm}
\usepackage{tgtermes}
\usepackage{natbib}
%\usepackage[subnum]{cases}
\usepackage[super]{nth}
\bibpunct{(}{)}{;}{a}{}{,}
\usepackage{amsmath,amssymb}
\usepackage{enumerate}
\usepackage{multicol}
\usepackage{tikz}
\usepackage[amssymb]{SIunits}
\usepackage{rotating}
\usepackage{enumitem}
\usepackage{geometry}
\geometry{total={8.5in,11in},
left=1in,right=1in,%
bindingoffset=0mm, top=1in,bottom=1in}
\usepackage[super]{nth}
\usepackage[
pdftitle={EFD: Exam 1, take-home}, 
pdfauthor={Jeremy Gibbs, University of Utah},
colorlinks=true,linkcolor=blue,urlcolor=blue,citecolor=blue,bookmarks=true,
bookmarksopenlevel=2]{hyperref}

\linespread{1.1}
\setlength{\parskip}{1em}
\setlength{\parindent}{0pt}
\newcommand{\linia}{\rule{\linewidth}{0.5pt}}
\makeatletter
\renewcommand{\maketitle}{
\begin{center}
\vspace{2ex}
{\huge \textsc{\@title}}
\vspace{1ex}
\\
\linia\\
ME EN 7710 \hfill Exam \#1 \hfill Take-Home Portion
\vspace{4ex}
\end{center}
}
\makeatother
%%%

% custom footers and headers
\usepackage{fancyhdr,lastpage}
\pagestyle{fancy}
\lhead{}
\chead{}
\rhead{}
%\lfoot{Assignment \textnumero{} 5}
\cfoot{}
\rfoot{Page \thepage~/~\pageref*{LastPage}}
\renewcommand{\headrulewidth}{0pt}
\renewcommand{\footrulewidth}{0pt}
%

%%%----------%%%----------%%%----------%%%----------%%%

\begin{document}

\title{Environmental Fluid Dynamics}

\maketitle

%-- Question 1 --%
\paragraph{1.) [30 points] Turbulence Kinetic Energy}
\begin{enumerate}[label=\alph*.)]
	
\item ~[18 points] Start with the provided turbulent momentum flux equation and derive the balance equation for turbulence kinetic energy (0.5$\overline{u_i^\prime u_i^\prime}$). Be sure to assume incompressibility.

\begin{align*}
\frac{\partial (\overline{u_k^\prime u_i^\prime})}{\partial t} + \overline u_j \frac{\partial (\overline{u_k^\prime u_i^\prime})}{\partial x_j} &= -\left[\overline{u_j^\prime u_i^\prime}\frac{\partial \overline u_k}{\partial x_j} + \overline{u_k^\prime u_j^\prime}\frac{\partial \overline u_i}{\partial x_j}\right] - \frac{\partial (\overline{u_k^\prime u_j^\prime u_i^\prime})}{\partial x_j} \\&\hphantom{=\ }+ \overline{u_k^\prime b^\prime}\delta_{i3} + \overline{u_i^\prime b^\prime}\delta_{k3}\\&\hphantom{=\ }-\left[ \frac{\partial(\overline{u_k^\prime \Pi^\prime})}{\partial x_i} + \frac{\partial(\overline{u_i^\prime \Pi^\prime})}{\partial x_k}- \overline{\Pi^\prime \left(\frac{\partial u_k^\prime}{\partial x_i} + \frac{\partial u_i^\prime}{\partial x_k}\right)}\right] +\nu \frac{\partial^2 (\overline{u_k^\prime u_i^\prime})}{\partial x_j^2} - 2\nu \overline{\frac{\partial u_k^\prime}{\partial x_j}\frac{\partial u_i^\prime}{\partial x_j}}
\end{align*}

\item ~[6 points] Identify the meaning of each of the terms in your final equation.
\item Simplify your turbulence kinetic energy equation for the following cases:
	\begin{itemize}
		\item ~[2 points] Horizontally homogeneous + no subsidence ($\overline{w}=0$)
		\item ~[2 points] Horizontally homogeneous + steady state
		\item ~[2 points] Horizontally homogeneous + steady state + no subsidence + neutral stability
	\end{itemize}
\end{enumerate}

%-- Question 2 --%
\vspace{-20pt}
\paragraph{2.) [20 points] Radiation}
\begin{enumerate}[label=\alph*.)]
\item ~[10 points] Calculate the total radiative flux for a black body surface at a temperature of $302\ \kelvin$, emitting radiation over the following wavelengths: $5\ \micro\metre$ to $40\ \micro\metre$. Feel free to integrate numerically. What type of radiation is this considered?
\item ~[10 points] Calculate the wavelength of maximum radiant energy.
\end{enumerate}

%-- Question 3 --%
\paragraph{3.) [20 points] Heat Budget}
\begin{enumerate}[label=\alph*.)]
\item[] In the afternoon, the incoming short wave radiation was measured at $50\ \metre$ above ground to be $750\ \watt\ \metre\rpsquared$. The layer cools at a rate of $dT/dt = 0.02\celsius/\text{day}$. Calculate the latent and sensible heat flux for the following environments (assume that the net longwave radiation is very small and that the flux into the submedium is negligible):
\item ~[10 points] a desert surface (albedo = 0.3, Bowen Ratio = 10)
\item ~[10 points] an irrigated crop surface (albedo=0.18, Bowen Ratio = 0.2)
\end{enumerate}
 
%-- Question 4 --%
\paragraph{4.) [30 points] Sensible, Latent, and Buoyancy Fluxes}
\begin{enumerate}[label=\alph*.)]
\item ~[20 points] The turbulent buoyancy flux $\overline{w^\prime b^\prime}$ can be defined as $\frac{g}{\overline{\theta_v}} \overline{w^\prime \theta_v^\prime}$. Consider just the $\overline{w^\prime \theta_v^\prime}$ portion of the equation. Using the definition of virtual temperature, Reynolds averaging, appropriate assumptions, and the relationships derived in class, derive an expression for $\overline{w^\prime \theta_v^\prime}$ in terms of the kinematic sensible heat
flux ($\overline{w^\prime T^\prime}$) and the kinematic latent heat flux ($\overline{w^\prime q^\prime}$). Please
show all work and clearly identify assumptions.

\item ~[10 points] Using your relationship from part (a), please explain the effect of moisture on the buoyancy flux.
\end{enumerate}

\end{document}