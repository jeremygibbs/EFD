\include{lecture_header}
\title{Environmental Fluid Dynamics: Lecture 21}
% colors
\usepackage[loadonly]{enumitem}
\newcommand{\ihat}{\boldsymbol{\hat{\imath}}}
\newcommand{\jhat}{\boldsymbol{\hat{\jmath}}}
\newcommand{\khat}{\boldsymbol{\hat{k}}}
\definecolor{colororange}{HTML}{E65100} % orange
\definecolor{colordgray}{HTML}{795548} % dark gray for note
\definecolor{colorhgray}{HTML}{212121} % heavy dark gray for normal text
\definecolor{colorgreen}{HTML}{009688} % green
\definecolor{colorwhite}{HTML}{FFFFFF} % background white
\definecolor{colorlgray}{HTML}{F5F3EE} % background light gray
\definecolor{colorblue}{HTML}{0277BB} % blue
\definecolor{colorred}{HTML}{CC0000} % red
\newcommand{\fontsizeone}{1.9em}
\usepackage{esvect}
\setbeamertemplate{caption}{\raggedright\insertcaption\par}
\newcommand{\framecard}[2][colorgreen]{
  {\setbeamercolor{background canvas}{bg=#1}
    \begin{frame}[plain]
    \vfill
    \begin{center}
     {#2}
    \end{center}
    \vfill
    \end{frame}
  }
}
\begin{document}

%----------------------------------------------------------------------------------------
%	TITLE & TOC SLIDES
%----------------------------------------------------------------------------------------

\begin{frame} 
  \titlepage
\end{frame}

%------------------------------------------------

\begin{frame}
\frametitle{Overview}
\tableofcontents
\end{frame}

%------------------------------------------------
\section{Monin-Obukhov Similarity Theory} %
%------------------------------------------------
\framecard[colorred]{{\color{white}\Huge MOST}}
%------------------------------------------------
\subsection{Recap}
%------------------------------------------------
\begin{frame}{Monin-Obukhov Similarity Theory Recap}
Recall from the previous lecture that we used Buckingham Pi theory to relate non-dimensionalized gradients to fluxes in the atmospheric surface layer.
~\\~\\
\textbf{MOST Assumptions}
\begin{itemize}
	\item flow is quasi-stationary and horizontally-homogeneous
	\item turbulent fluxes are constant with height within the ASL.
	\item molecular exchanges are small compared to turbulent exchanges.
	\item rotational effects are neglected.
	\item influence of surface roughness, boundary-layer depth, and geostrophic wind are accounted for by $\tau_w/\rho$.
\end{itemize}
\end{frame}
%------------------------------------------------
\begin{frame}{Monin-Obukhov Similarity Theory Recap}

\textbf{Scaling variables}
\begin{align*}
	u_* &\simeq (-\overline{w^\prime u^\prime})^{1/2} \\
	\theta_* &= -(\overline{w^\prime \theta^\prime})/u_*\\
	\theta_{v*} &= -(\overline{w^\prime \theta_v^\prime})/u_*\\
	q_* &= -(\overline{w^\prime q^\prime})/u_*\\
	b_* &= -(\overline{w^\prime b^\prime})/u_*\\
	L&=-u_*^3/(\kappa B_0) = u_*^2 / \kappa b_*
	\end{align*}
	where $L$ is the Obukhov Length, which describes the characteristic height of the sublayer of dynamic turbulence.
\end{frame}

%------------------------------------------------
\begin{frame}{Monin-Obukhov Similarity Theory Recap}

\textbf{Similarity Theory}
\begin{itemize}
	\item Similarity Theory showed that mean flow variables or average turbulence quantities, when normalized by $z, L, u_*, \theta_*$, etc., are functions of $\zeta = z/L$ only!
	\item $\zeta$ helps determine the relative importance of buoyancy versus shear effects, which makes it akin to the Richardson number (Ri).
	\begin{itemize}
		\item $z \gg L$, buoyancy dominates
		\item $z \ll L$, shear dominates
	\end{itemize}
\end{itemize}
\end{frame}
%------------------------------------------------
\begin{frame}{Monin-Obukhov Similarity Theory Recap}

\textbf{Similarity Theory}
\begin{itemize}
	\item We found these flux-profile relationships
	\begin{align*}
		\frac{\kappa z}{u_*} \frac{\partial \overline{u}}{\partial z} = \phi_m \left(\zeta \right) & \quad & \frac{\kappa z}{\theta_*} \frac{\partial \overline{\theta}}{\partial z} &= \phi_h \left(\zeta \right)\\
		\frac{\kappa z}{\theta_{v*}} \frac{\partial \overline{\theta_v}}{\partial z} = \phi_v \left(\zeta \right) & \quad & \frac{\kappa z}{b_*} \frac{\partial \overline{b}}{\partial z} &= \phi_b \left(\zeta \right)\\
			\frac{\kappa z}{q_{*}} \frac{\partial \overline{q}}{\partial z} = \phi_q \left(\zeta \right)
		\end{align*}
		\item Where $\phi$ terms are universal functions of $z/L$ and we often assume $\phi_h = \phi_v = \phi_b = \phi_q$.
\end{itemize}
\end{frame}
%------------------------------------------------
\begin{frame}{Monin-Obukhov Similarity Theory Recap}

\textbf{Similarity Theory}
\begin{itemize}
	\item We chose the empirical forms of the similarity functions as derived by Dyer (1974).
	\begin{align*}
		\text{neutral} & \quad & \phi_m &= 1 & \quad  \phi_h &= 1\\
		\text{unstable} & \quad & \phi_m &= \left(1 - 16\zeta\right)^{-1/4} & \quad  \phi_h &= \left(1 - 16\zeta\right)^{-1/2}\\
		\text{stable} & \quad & \phi_m &= 1 + 5\zeta & \quad  \phi_h &= 1 + 5\zeta
		\end{align*}
		\item In totality, MOST allows us to determine turbulent fluxes from the mean gradients (or gradients from fluxes)
\end{itemize}
\end{frame}
%------------------------------------------------
\subsection{Relationships}
%------------------------------------------------
\begin{frame}{Monin-Obukhov Similarity Theory Relationships}

\begin{itemize}
	\item Let's relate these functions to Ri
	\item The flux Richardson number and gradient Richardson number are, respectively:
	$$\mathrm{Ri_f} = \frac{\overline{w^\prime b^\prime}}{\overline{w^\prime u^\prime}\partial \overline{u}/\partial z} \quad \text{and} \quad \mathrm{Ri} = \frac{\partial \overline{b}/\partial z}{(\partial \overline{u}/\partial z)^2}$$
	\item Recall our scales: $-\overline{w^\prime u^\prime} = u_*^2$ and $-\overline{w^\prime b^\prime} = u_*b_*$.
	\item And use our flux-profile relationships:
	$$\frac{\kappa z}{u_*} \frac{\partial \overline{u}}{\partial z} = \phi_m \quad \text{and} \quad \frac{\kappa z}{b_*} \frac{\partial \overline{b}}{\partial z} = \phi_b$$
	\item With the Obukhov Length
	$$L=-u_*^3/(\kappa B_0) = u_*^2 / \kappa b_*$$
\end{itemize}
\end{frame}
%------------------------------------------------
\begin{frame}{Monin-Obukhov Similarity Theory Relationships}

\begin{itemize}
	\item Flux Richardson number
	\begin{align*}
	 \mathrm{Ri_f} &= \frac{\overline{w^\prime b^\prime}}{\overline{w^\prime u^\prime}\partial \overline{u}/\partial z} = \frac{- u_* b_*}{-u_*^2 \partial \overline{u}/\partial z} = \frac{u_* b_* \kappa z}{u_*^3 \phi_m} = \frac{b_* \kappa z}{u_*^2 \phi_m} = \frac{z}{L \phi_m}\\
	 \Aboxed{\mathrm{Ri_f} &= \zeta \phi_m^{-1}}
	 \end{align*}
	 \item Gradient Richardson number
	 \begin{align*}
	 	\mathrm{Ri} &= \frac{\partial \overline{b}/\partial z}{(\partial \overline{u}/\partial z)^2} = \frac{\cfrac{b_*}{\kappa z}\phi_h}{\cfrac{u_*^2}{(\kappa z)^2} \phi_m^2} = \frac{\kappa z b_* \phi_h}{u_*^2 \phi_m^2} = \frac{z\phi_h}{L \phi_m^2}\\
	 	\Aboxed{\mathrm{Ri} &= \zeta \frac{\phi_h}{\phi_m^{2}}}
	 \end{align*}
\end{itemize}
\end{frame}
%------------------------------------------------
\begin{frame}{Monin-Obukhov Similarity Theory Relationships}

\begin{itemize}
	\item Let's use K-theory to derive expressions that relate similarity functions to the turbulent Prandtl and Schmidt numbers.
	\begin{align*}
		-K_m\frac{\partial \overline{u}}{\partial z} &= \overline{w^\prime u^\prime} &\quad  -K_h\frac{\partial \overline{b}}{\partial z} &= \overline{w^\prime b^\prime}\\
		K_m\frac{\partial \overline{u}}{\partial z} &= u_*^2 &\quad  K_h\frac{\partial \overline{b}}{\partial z} &= u_*b\\
		K_m &= \frac{u_*^2}{\cfrac{u_*}{\kappa z}\phi_m} &\quad K_h &= \frac{u_*b_*}{\cfrac{b_*}{\kappa z}\phi_h}\\
		\Aboxed{K_m &= \frac{\kappa z u_*}{\phi_m}} &\quad \Aboxed{K_h &= \frac{\kappa z u_*}{\phi_h}}
	\end{align*}
\end{itemize}
\end{frame}
%------------------------------------------------
\begin{frame}{Monin-Obukhov Similarity Theory Relationships}

\begin{itemize}
	\item The Prandtl and Schmidt numbers are defined as:
	$$\mathrm{Pr} = \nu/\nu_h \quad \text{and} \quad \mathrm{Sc} = \nu/\nu_q$$
	\item Analogously, we define their turbulent versions:
	$$\mathrm{Pr_t} = K_m/K_h \quad \text{and} \quad \mathrm{Sc_t} = K_m/K_q$$
	\item Thus,
	\begin{align*}
	\mathrm{Pr_t} = \frac{\cfrac{\kappa z u_*}{\phi_m}}{\cfrac{\kappa z u_*}{\phi_h}} = \frac{\phi_h}{\phi_m} &\qquad \mathrm{Sc_t} = \frac{\cfrac{\kappa z u_*}{\phi_m}}{\cfrac{\kappa z u_*}{\phi_q}} = \frac{\phi_q}{\phi_m}
	\end{align*}
	\item Recall, however, that we assume $\phi_q \approx \phi_h$, thus
	$$\boxed{\mathrm{Pr_t} = \mathrm{Sc_t} = \frac{\phi_h}{\phi_m}}$$
\end{itemize}
\end{frame}
%------------------------------------------------
\begin{frame}{Monin-Obukhov Similarity Theory Relationships}

\begin{itemize}
	\item Let's relate $\mathrm{Pr_t}$ and $\mathrm{Sc_t}$ to $\mathrm{Ri_f}$ and $\mathrm{Ri}$ :
	$$\mathrm{Ri} = \zeta \frac{\phi_h}{\phi_m^2} = \mathrm{Ri_f}\frac{\phi_h}{\phi_m}=\mathrm{Ri_f}\mathrm{Pr_t}=\mathrm{Ri_f}\mathrm{Sc_t}$$
	or
	$$\boxed{\mathrm{Pr_t} = \mathrm{Sc_t} = \frac{\mathrm{Ri}}{\mathrm{Ri_f}}}$$
\end{itemize}
\end{frame}
%------------------------------------------------
\begin{frame}{Monin-Obukhov Similarity Theory Relationships}

\begin{itemize}
	\item Consider \textit{unstable} conditions using Dyer's functions
	\begin{align*}
	& \phi_m = \left(1 - 16\zeta\right)^{-1/4} & \quad  \phi_h = \left(1 - 16\zeta\right)^{-1/2}
	\end{align*}
	\begin{align*}
		\mathrm{Ri} &= \zeta \frac{\phi_h}{\phi_m^2} = \zeta \leq 0\\
		\mathrm{Ri_f} &= \zeta \phi_m^{-1} = \zeta(1-16\zeta)^{1/4} \leq 0
	\end{align*}
\end{itemize}
\end{frame}
%------------------------------------------------
\begin{frame}{Monin-Obukhov Similarity Theory Relationships}

\begin{itemize}
	\item Consider \textit{stable} conditions using Dyer's functions
	\begin{align*}
	& \phi_m = 1 + 5\zeta & \quad  \phi_h = 1 + 5\zeta
	\end{align*}
	\begin{align*}
		\mathrm{Ri} &= \zeta \frac{\phi_h}{\phi_m^2} = \zeta \phi_m^{-1} = \mathrm{Ri_f} = \frac{\zeta}{5\zeta + 1} \geq 0 \\
	\end{align*}
	\item We can rearrange as
	$$\zeta = \frac{\mathrm{Ri}}{1 -5\mathrm{Ri}} \quad 0\leq \mathrm{Ri} < 0.2$$
	\item Note that for $\mathrm{Ri} = 0.2$, $\zeta \rightarrow \infty\ (L\rightarrow 0)$. This means that there is no turbulence beyond this value. Thus, the Dyer functions point to $\mathrm{Ri_c} = 0.2$.
\end{itemize}
\end{frame}
%------------------------------------------------
\subsection{Surface-Layer Logarithmic Wind Profile}
\begin{frame}{Surface-Layer Logarithmic Wind Profile}

\begin{itemize}
	\item Consider the case of neutral stratification ($\phi_m = 1$)
	\begin{align*}
		\frac{\kappa z}{u_*} \frac{\partial \overline{u}}{\partial z} &= 1\\
		\frac{\partial \overline{u}}{\partial z} &= \frac{u_*}{\kappa z} \quad \text{now integrate}\\
		\Aboxed{u &= \frac{u_*}{\kappa}\ln{z} + C}
	\end{align*}
	where $C$ is a constant of integration.
	\item This describes the famous logarithmic wind profile in the atmospheric surface layer.
	\item Recall that wind should adhere to no-slip conditions ($u=0$) at the surface. However, notice that there is discontinuity at $z=0$. This points to the fact that the flow becomes laminar for very small $z$ and brings about the concept of surface roughness.
\end{itemize}
\end{frame}
%------------------------------------------------
\subsection{Aerodynamically Smooth and Rough Surfaces}
\begin{frame}{Aerodynamically Smooth and Rough Surfaces}

\begin{itemize}
	\item If we take $u_*$ as the velocity scale and $\delta_\ell$ as the length scale of turbulence in the ASL, then the Reynolds number criterion for laminarization of the flow close to the surface (wall) is
	$$\mathrm{Re_\delta} = \frac{u_* \delta_\ell}{\nu} \sim 1$$
	where $\nu$ is kinematic viscosity
	\item Thus, turbulence does not exist at distances from the wall of the order and less than $\delta_\ell \sim \nu/u_*$ (note: the oft-used viscous wall units are defined as $z^+ = z/\delta_\ell$ and $u^+ = \overline{u}/u_*$)
	\item Experimental data suggest that $\delta_\ell \sim 5\ \nu/u_*$, where the layer defined with this depth is called the \textit{viscous sublayer}. 
\end{itemize}
\end{frame}
%------------------------------------------------
\begin{frame}{Aerodynamically Smooth and Rough Surfaces}

\textbf{Aerodynamically Smooth}
\begin{itemize}
	\item If roughness elements of characteristic size $z_r$ are deployed in the viscous sublayer and $z_r \ll \delta_l$, then the surface is \textit{aerodynamically smooth}.
	\item Lab data shows that surfaces are smooth for $z_r \leq 5 \nu/u_*$.
	\item For the atmosphere, this corresponds to $z_r \lesssim 1\ \milli\metre$. 
	\item However, most elements in the ASL are larger than $1\ \milli\metre$.
	\item Thus, most surfaces in the ASL are aerodynamically rough (exceptions: ice, mudflats, snow, water under light wind).
\end{itemize}

\textbf{Aerodynamically Rough}
\begin{itemize}
	\item The surface is \textit{aerodynamically rough} for $z_r \gg \delta_l$.
	\item Lab data shows that surfaces are rough for $z_r \geq 75 \nu/u_*$.
\end{itemize}
\end{frame}
%------------------------------------------------
\begin{frame}{Aerodynamic Surface Roughness Length}

\begin{itemize}
	\item In the case of a smooth surface, a turbulence flow regime represented by a logarithmic profile us possible at a height above $\nu/u_*$ (well above surface roughness elements).
	\item In the case of a rough surface, the flow is already turbulent in the near vicinity of surface roughness elements. Measurements show that $u=0$ at some level close to $z_r$ (actually just below).
	\item Let's introduce the idea of the surface roughness length.
\end{itemize}
\end{frame}
%------------------------------------------------
\begin{frame}{Aerodynamic Surface Roughness Length}

\begin{itemize}
	\item Recall the generic log-law profile:
	$$u = \frac{u_*}{\kappa}\ln{z} + C$$
	\item We will introduce a reference level $z_0$ where $u=0$, defined through
	$$C = -\left(\frac{u_*}{\kappa}\right)\ln{z_0}$$
	\item This leads to the neutral log-law profile
	$$\boxed{u = \frac{u_*}{\kappa}\ln{\frac{z}{z_0}}}$$
	where $z_0$ is called the aerodynamic surface roughness length (or surface roughness length) for momentum.
\end{itemize}
\end{frame}
%------------------------------------------------
\begin{frame}{Mean Flow Above a Smooth Surface}

\begin{itemize}
	\item For smooth surfaces, $z_0$ defines the lower asymptotic limit of the logarithmic wind profile, below which the mean flow velocity is no longer a characteristic of turbulence.
	\item We can rearrange the neutral log-law expression and scale height by $\delta_\ell = \nu/u_*$
	\begin{align*}
		\frac{u}{u_*} &= \frac{1}{\kappa} \ln{\frac{z}{\nu / u_*}} + \frac{1}{\kappa} \ln{\frac{\nu / u_*}{z_0}}\\
		\frac{u}{u_*} &= \frac{1}{\kappa}  \ln{\frac{z}{\nu / u_*}} + C_s
	\end{align*}
	where $$C_s = \frac{1}{\kappa} \ln{\frac{\nu / u_*}{z_0}}.$$
	Lab data suggest that $C_s \approx 5$
\end{itemize}
\end{frame}
%------------------------------------------------
\begin{frame}{Mean Flow Above a Smooth Surface}

\begin{itemize}
	\item The final form is given by
	$$\boxed{\frac{u}{u_*} = \frac{1}{\kappa}  \ln{\frac{zu_*}{\nu}} + 5} \quad \text{or} \quad \boxed{u^+ = \frac{1}{\kappa}  \ln{z^+} + 5}$$
	\item We can also approximate $z_0$:
	\begin{align*}
		C_s &= \frac{1}{\kappa} \ln{\frac{\nu / u_*}{z_0}}\\
		\frac{\nu / u_*}{z_0} &= e^{\kappa C_s}\\
		\Aboxed{z_0 &= e^{-\kappa C_s} \frac{\nu}{u_*} \approx 0.1 \frac{\nu}{u_*}}
	\end{align*}
	Or in other words, the surface roughness length for a smooth surface is approximately 10\% of the viscous sublayer depth.
\end{itemize}
\end{frame}
%------------------------------------------------
\begin{frame}{Mean Flow Above a Smooth Surface}

\begin{itemize}
	\item The final form is given by
	$$\boxed{\frac{u}{u_*} = \frac{1}{\kappa}  \ln{\frac{zu_*}{\nu}} + 5} \quad \text{or} \quad \boxed{u^+ = \frac{1}{\kappa}  \ln{z^+} + 5}$$
	\item We can also approximate $z_0$:
	\begin{align*}
		C_s &= \frac{1}{\kappa} \ln{\frac{\nu / u_*}{z_0}}\\
		\frac{\nu / u_*}{z_0} &= e^{\kappa C_s}\\
		\Aboxed{z_0 &= e^{-\kappa C_s} \frac{\nu}{u_*} \approx 0.1 \frac{\nu}{u_*}}
	\end{align*}
	Or in other words, the surface roughness length for a smooth surface is approximately 10\% of the viscous sublayer depth.
\end{itemize}
\end{frame}
%------------------------------------------------
\begin{frame}{Mean Flow Above a Rough Surface}

\begin{itemize}
	\item For rough surfaces, $z_0$ is directly interpreted as the level where mean flow velocity vanishes. So,
	$$u = \frac{u_*}{\kappa}\ln{\frac{z}{z_0}} \quad \text{where } u=0 \text{ at } z=z_0 $$
	\item In the real world, $z_0$ is a complex function of surface geometry, involving $z_r$ as one of many parameters.
	\item Generally, $z_0$ increases with increasing $z_r$.
\end{itemize}
\end{frame}
%------------------------------------------------
\begin{frame}{Mean Flow Above a Rough Surface}

\begin{itemize}
	\item In reality, there is no real consistent average velocity observed in a flow down to $z_0$ (below $z_r$).
	\item The velocity field obeys the log-law only at some distance $z\gg z_0$ above the surface.
	\item In this sense, $z_0$ is also the asymptotic limit of the logarithmic velocity profile.
	\item In order to make more applicable, we introduce the concept of the displacement height $d$.
	$$u = \frac{u_*}{\kappa}\ln{\frac{z-d}{z_0}} \quad \text{where } u=0 \text{ at } z=z_0+d$$
	\item Far above the displaced height ($z\gg d$), $d$ is ignored
	$$u = \frac{u_*}{\kappa}\ln{\frac{z-d}{z_0}} = u = \frac{u_*}{\kappa}\ln{\frac{z/d-1}{z_0/d}} \approx u = \frac{u_*}{\kappa}\ln{\frac{z}{z_0}}$$
\end{itemize}
\end{frame}
%------------------------------------------------
\begin{frame}{$\mathbf{z_0}$ Parameterizations for Sand, Snow, Water}

\textbf{Snow, Sand}
\begin{itemize}
	\item $z_0$ for snow/sand increases with increasing wind speed.
	\item As wind speed increases, the material moves more actively and transports more effectively away from the surface.
	\item Empirical expression:
	$$z_0 = \frac{\alpha_s u_*^2}{g}$$
	where $\alpha_s = 0.016$ and $u_* > u_{*t}$. Here, $u_{*t} \approx 0.12\ \metre\ \reciprocal\second$ is a threshold frictions velocity. In the rough wall case, $z_0$ may be considered constant for snow/sand when $u_*<u_{*t}$.
\end{itemize}
\end{frame}
%------------------------------------------------
\begin{frame}{$\mathbf{z_0}$ Parameterizations for Sand, Snow, Water}

\textbf{Water}
\begin{itemize}
	\item Wind generates waves on a water's surface.
	\item Waves occur within a broad range of geometric parameters (heights/lengths).
	\item Waves are generated and grow due to many physical mechanism, such as wave age, fetch, depth of the water body, and wind velocity (in terms of $u_*$).
	\item Roughness of wavy water is primarily determined by the steepest waves, rather than the longest.
\end{itemize}
\end{frame}
%------------------------------------------------
\begin{frame}{$\mathbf{z_0}$ Parameterizations for Sand, Snow, Water}

\textbf{Water}
\begin{itemize}
	\item The shortest waves are capillary waves, with amplitudes/lengths $\mathcal{O}(1\ \milli\metre$).
	\item Water is typically considered aerodynamically smooth if $\mathrm{Re_*} \ll 1$, so if we estimate $\mathrm{Re_*} = z_0 u_* / \nu \approx 0.1$, then $z_0 = m_s\left(\nu/u_*\right)$, where $m_s \approx 0.1$.
	\item Water is fully rough when $\mathrm{Re_*} \gg 1$. In this case we use $z_0 = \alpha_c u_*^2/g$, where $\alpha_c$ is the Charnock ``constant'', which ranges from $0.01-0.035$ (typically $0.014-0.019)$.
\end{itemize}
\end{frame}
%------------------------------------------------
\begin{frame}{$\mathbf{z_0}$ for Temperature and Moisture}

\begin{itemize}
	\item Boundary conditions for temperature and moisture at the underlying surface are formulated based on notions of their roughness lengths.
	$$\theta = \theta_s \text{ at } z_{0\theta} \quad \text{and} \quad q = q_s \text{ at } z_{0q}$$
	where $z_{0\theta}$ and $z_{0q}$ are interpreted as the levels where $\theta$ and $q$ reach their surface values $\theta_s$ and $q_s$, respectively.
\end{itemize}
\end{frame}
%------------------------------------------------
\begin{frame}{$\mathbf{z_0}$ for Temperature and Moisture}

\begin{itemize}
	\item The physical nature of transport mechanisms for momentum, heat, and moisture differ significantly.
	\item e.g., pressure fluctuations are important to the transport of momentum, bu do not directly affect heat and moisture.
	\item Thus, there is no physical basis to expect that $z_0$ and $z_{0\theta},z_{0q}$ should be the same, or even close.
	\item There are experimental indications of similarity between heat and moisture, so $z_{0\theta} \sim z_{0q}$.
\end{itemize}
\end{frame}
%------------------------------------------------
\begin{frame}{Parameterizing the Relationships between $\mathbf{z_0}$ and $\mathbf{z_{0\theta},z_{0q}}$}

\begin{itemize}
	\item The number of roughness parameters needed is reduced by parameterizing relationships between them.
	\item Commonly, $z_0/z_{0\theta}$ and $z_0/z_{0q}$ are parameterized based on the assumption that $\theta$ and $q$ are logarithmic close to the surface.
	$$\theta(z) = \theta_s + \frac{\theta_*}{\kappa} \ln{\frac{z}{z_{0\theta}}} \quad \text{and} \quad q(z) =q_s + \frac{q_*}{\kappa} \ln{\frac{z}{z_{0q}}}$$
	thus,
	$$\delta \theta = \theta(z_0) - \theta_s = \frac{\theta_*}{\kappa} \ln{\frac{z}{z_{0\theta}}} \quad \text{and} \quad \delta q = q(z_0) - q_s = \frac{q_*}{\kappa} \ln{\frac{z}{z_{0q}}}$$
	\item Experimental data suggest that $\ln(z_0/z_{0\theta})$ and $\ln(z_0/z_{0q})$ may be functions of $\mathrm{Re_*} = z_0 u_*/\nu$ for rough surfaces.
\end{itemize}
\end{frame}
%------------------------------------------------
\begin{frame}{Parameterizing the Relationships between $\mathbf{z_0}$ and $\mathbf{z_{0\theta},z_{0q}}$}

\begin{itemize}
	\item \textbf{Rough}
	$$\frac{1}{\kappa}\ln{\frac{z_0}{z_{0\theta}}} = 6.2\ \mathrm{Re_*}^{1/4} - 5 \quad \text{and} \quad \frac{1}{\kappa}\ln{\frac{z_0}{z_{0q}}} = 5.7\ \mathrm{Re_*}^{1/4} - 5$$
	\item \textbf{Smooth}
	$$\frac{1}{\kappa}\ln{\frac{z_0}{z_{0\theta}}} = 13.6\ \mathrm{Pr}^{2/3} - 12 \quad \text{and} \quad \frac{1}{\kappa}\ln{\frac{z_0}{z_{0q}}} = 13.6\ \mathrm{Sc}^{2/3} - 12$$
	\item Typical ASL values for $\mathrm{Pr}$ and $\mathrm{Sc}$ are $0.71$ and $0.6$, respectively, for smooth surfaces.
	\item Thus, $z_0/z_{0\theta} = 0.5$ and $z_0/z_{0q} = 0.3$ (i.e., the roughness lengths for heat and moisture are typically larger than that for momentum).
\end{itemize}
\end{frame}
%------------------------------------------------

\end{document}

